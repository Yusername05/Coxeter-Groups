% Mathematics Second Year Research Projects
% LaTeX Template
% Version 1.0 (31/01/24)
%
% This template has been adapted from: https://www.overleaf.com/latex/templates/imperial-college-report-template/wncnzptkhnbc
\documentclass[a4paper,11pt, titlepage]{article}

\usepackage{style/header}
\usepackage{setspace}
%\doublespacing

\newcommand{\reporttitle}{Reflection Groups}
\newcommand{\reportauthorA}{Jialin Li (CID: 02209592)}
\newcommand{\reportauthorB}{Yu Coughlin (CID: 02386349)}
\newcommand{\reportauthorC}{Student name 3 (CID: -------------)}
\newcommand{\reportauthorD}{Student name 4 (CID: -------------)}
\newcommand{\reportauthorE}{Heng Wang (CID: 02395274)}
\newcommand{\supervisor}{Alessio Corti}

% To compile this file locally use Recipe: pdflatex -> bibtex -> pdflates * 2
\begin{document}

\begin{titlepage}
\subfile{title/title.tex}
\end{titlepage}

\begin{abstract}
Type your abstract here. The abstract is a summary of the contents of the project. It should be brief but informative, and
should avoid technicalities as far as possible.
\end{abstract}

\tableofcontents
\clearpage

\section{Introduction}
The introduction should attempt to set your work in the context of other work done in the field. It
should demonstrate that you are aware of what you are doing, and how it relates to other work
(with references). It should also provide an overview of the contents of the project. You should
highlight your individual contributions and any novel result: which of the calculations, theorems,
examples, proofs, conjectures, codes etc. are your own? This is how you cite a reference in the bibliography\cite{Humphreys1990}. All of the commands and formatting are in ./style/header.sty

\section{Dihedral groups}
\subfile{polytopes/polytopes.tex}

\section{Classification of finite reflection groups}\label{Classification}
\subfile{classification/classification.tex}

\section{Uniform polytopes}
\subfile{uniform/uniform.tex}

\section{Tits' word problem}
\subfile{word/word.tex}

\section{Euclidean Reflection Groups}
\subfile{decidability/decidability.tex}

\appendix
\section{Model-Theoretic Preliminaries}\label{Model Theory}
\subfile{decidability/models.tex}

\bibliography{bibliography.bib}

\end{document}