% Mathematics Second Year Research Projects
% LaTeX Template
% Version 1.0 (31/01/24)
%
% This template has been adapted from: https://www.overleaf.com/latex/templates/imperial-college-report-template/wncnzptkhnbc
\documentclass[a4paper,11pt, titlepage]{article}

\usepackage{style/header}
\usepackage{setspace}
%\doublespacing

\newcommand{\reporttitle}{Reflection Groups}
\newcommand{\reportauthorA}{Jialin Li (CID: 02209592)}
\newcommand{\reportauthorB}{Yu Coughlin (CID: 02386349)}
\newcommand{\reportauthorC}{Student name 3 (CID: -------------)}
\newcommand{\reportauthorD}{Student name 4 (CID: -------------)}
\newcommand{\reportauthorE}{Heng Wang (CID: 02395274)}
\newcommand{\supervisor}{Alessio Corti}

% To compile this file locally use Recipe: pdflatex -> bibtex -> pdflates * 2
\begin{document}

\begin{titlepage}
\subfile{title/title.tex}
\end{titlepage}

\begin{abstract}

\end{abstract}

\tableofcontents
\clearpage

\section{Introduction}
In lectures, we have seen how to classify finitely generated abelian groups. A main objective of this report is to classify finite reflection groups.

Recall that a reflection is a linear operator on $\mathbb{R}^n$ which sends some nonzero vector $\mathbf{u}$ to its negative while fixing pointwise the hyperplane $H_{\mathbf{u}}$ orthogonal to $\mathbf{u}$. With the usual inner product, the reader can check that $s_{\mathbf{u}}(\mathbf{v}) = \mathbf{v} - 2\frac{\mathbf{v} \cdot \mathbf{u}}{\mathbf{u} \cdot \mathbf{u}}\mathbf{u}$.

We will start by looking at the lower-dimensional reflection groups and proceed to more general ones, introducing some related terminology and ideas along the way.
Classification of reflection groups not only describes the symmetries of regular polytopes but also helps understand more complex algebraic structures. 




We then focus on affine reflection groups. By relating them to finite reflection groups, we show more finiteness properties of them, which gives us a structural theorem on Euclidean reflection groups. From this theorem, we show this structure can be interpreted in Presburger arithmetic, hence deduce the decidability of Euclidean reflection groups. 

\section{Dihedral groups}
\subfile{polytopes/polytopes.tex}

\section{Classification of finite reflection groups}\label{Classification}
\subfile{classification/classification.tex}

\section{Uniform polytopes}
\subfile{uniform/uniform.tex}

\section{Tits' word problem}\label{word}
\subfile{word/word.tex}

\section{Euclidean Reflection Groups}
\subfile{decidability/decidability.tex}

\appendix
\section{Model-Theoretic Preliminaries}\label{Model Theory}
\subfile{decidability/models.tex}

\bibliography{bibliography.bib}

\end{document}