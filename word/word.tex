\documentclass[../main.tex]{subfiles}
\usepackage{graphicx} % Required for inserting images
\usepackage{amsfonts} % For \mathbb
\usepackage[a4paper, left=1in, right=1in, top=1in, bottom=1in]{geometry}
\usepackage{amsmath, amsthm}
\begin{document}

\setcounter{subsection}{-1}
\subsection{Abstract}

Recalling the word problem for a presentation $\langle S\mid R\rangle$ of a group $G$, we hope to find an algorithm to reduce any word in $W$ into a reduced word. In this paper, we totally solved this problem for Coxeter groups, and actually for all system with exchange condition, which is called Tit's theorem.

\noindent The proof is divided into several steps:

1.Every Coxeter group satisfies the exchange condition.

2.For a system with exchange condition, when do two reduced words represent the same element.

3.How to reduce a word in a system with exchange condition.

The first part of the proof is the hardest, and we focus on the most.

We will talk more on the exchange condition as the fourth part, even though part 1 is enough to prove Tit's Theorem, exchange condition show deep insight of Coxeter Groups.

\subsection{Exchange Condition}
\textbf{Definition 1.1} A system is a tuple $(W,S)$, where $W$ is a group and $S$ is a set of generators of $G$, such that $S=S^{-1}$ and $1\notin S$.

\vspace{\baselineskip}
\noindent \textbf{Definition 1.2} For $w \in W$, the sequence $\mathbf{s}=(s_1,...,s_q)$,where $s_i \in S$ for all $i\in \mathbb{N}$, is called a reduced representation of $w$ if and only if $w=s_1s_2...s_q$ $(*)$ and $q$ is the smallest among all such sequences satisfies $(*)$. We define $l(w)=q$ and call it the length of $w$.

\vspace{\baselineskip}
\noindent \textbf{Proposition 1.3} For $w,w'\in W$, we have the following facts:

\indent (1) $l(ww')\le l(w)+l(w')$

\indent (2) $l(w^{-1}) = l(w)$

\indent (3) $|l(w)-l(w')| \le l(ww'^{-1})$

\noindent \textbf{Proof:} (1) and (2) are trivial. From (1), we see $l(w) \le l(ww'^{-1})+l(w')$ and $l(w') \le l(w'w^{-1})+l(w)$. From (2), we see $l(w'w^{-1})=l(ww'^{-1})$. Putting these together gives us (3). \qed

\vspace{\baselineskip}
\noindent \textbf{Definition 1.4} If a system $(W,S)$ satisfies the following conditions:

\indent (1)Every element in $S$ has order 2.

\indent (2) Say $m(s,t)$ is the order of $st$, and define $I = \{ (s,t) \mid s,t \in S,\ m(s,t) < \infty \}$, then $\langle S \mid (st)^{m(s,t)}=1,(s,t)\in I \rangle$ is a presentation of $W$.

Then we say $(W,S)$ is a Coxeter System and W is a Coxeter group.

\vspace{\baselineskip}
\noindent \textbf{Remark 1.5} A Coxeter System $(W,S)$ satisfies the following universal property: For any group $G$ and map $\psi:S \rightarrow G$ such that $\psi(st) ^{m(st)}=e_G$,$\forall(s,t) \in I$, there exist an unique $\phi \in Hom(W,G)$ such that $\psi=\phi \circ l$. Where $I$ is defined as in Definition 1.4, and $l$ is the natural inclusion from $S$ to $W$. Note that this is just the composition of universal properties of free groups and quotient groups.

\vspace{\baselineskip}
\noindent \textbf{Example 1.6} By universal property of Coxeter System, the map $s \mapsto -1$, $\forall s \in S$ extends to a homomorphism $\epsilon :W \rightarrow \{-1,1\}$ such that $\epsilon(w)=(-1)^{l(w)}$. We call $\epsilon(w)$ the sign of $w$. 

\vspace{\baselineskip}
We want to find some invariant among representations of the same word, hence make the following definition.

\vspace{\baselineskip}
\noindent \textbf{Definition 1.7} For Coxeter System $(W,S)$, say $T$ the union of conjugacy classes of all $s \in S$. For sequence $\mathbf{s}=(s_1,...,s_q)$, where $s_i \in S,\forall i$. Define $\Phi(\mathbf{s})=(t_1,...,t_q)$ such that $t_j=(s_1...s_{j-1})s_j(s_1...s_{j-1})^{-1}$. For $t\in T$, define $n(\mathbf{s},t):= \# \{1\le j \le q \mid t_j=t\}$. Finally, define $R=\{\pm 1\} \times T$.

\noindent \textbf{Remark} Note that we have $s_1...s_j=t_j...t_1$.

\vspace{\baselineskip}
\noindent \textbf{Lemma 1.8}  For Coxeter System $(W,S)$, we have the following facts:

\indent (1) For $w \in W$ and $t \in T$, $\forall \mathbf{s}=(s_1,...,s_q)$ such that $w=s_1...s_q$, $(-1)^{n(\mathbf{s},t)}$ has the same value. We call this value $\eta(w,t)$.

\indent (2) For $w \in W$, consider the map $U_w:R \rightarrow R$ defined as $U_w(\epsilon,t)=(\epsilon \eta (w,t),wtw^{-1})$, then $w \mapsto U_w$ is a homomorphism from $W$ to the permutation group of $R$. 

\vspace{0.5\baselineskip}
\noindent \textbf{Proof:} For $s \in S$, we define $U_s:R \rightarrow R$ such that $(\epsilon,t) \mapsto (\epsilon(-1)^{\delta(\mathbf{s},t)},sts^{-1})$, where $\delta_{s,t}$ is the Kronecker Symbol. For sequence $\mathbf{s}=(s_1,...,s_q)$ in $S$, say $w=s_q...s_1$, $U_s=U_{s_q}\circ...\circ U_{s_1}$.

 We prove $U_{\mathbf{s}}(\epsilon,t)=(\epsilon(-1)^{n(\mathbf{s},t)},wtw^{-1})$ by induction. $(*)$

 In case $q=0$ or $1$, the proof is trivial. For $q >1$, by induction hypothesis, say $\mathbf{s'}=(s_1,...,s_{q-1})$,$w'=s_{q-1}...s_1$, then $U_{\mathbf{s'}}=U_{s_q}(\epsilon(-1)^{n(\mathbf{s'},t)},w'tw^{-1})$. Hence it only remains to prove that $n(\mathbf{s},t)=\delta_{s_q,w'tw^{-1}}+n(\mathbf{s'},t)$. Notice that $\Phi(\mathbf{s})=(\Phi(\mathbf{s'}),w'^{-1}s_qw')$, so the proof is trivial.

 We now prove the fact that $s\mapsto U_s$ can be extended to a homomorphism from $W$. By the universal property of Coxeter System, it saffice to prove that $\forall s,s' \in S$ such that $m(s,s') < \infty$, $(U_s \circ U_{s'})^{m(s,s')}=1$. Define $\mathbf{s}=(s_1,...,s_{2m(s,s')})$ such that $s_i:=s$ for odd $i$ and $s_i:=s'$ for even $i$, so it saffice to prove $U_{\mathbf{s}}=Id$. Notice that here we have $t_j=(ss')^{j-1}s$, so $t_i \ne t_j$ for all $1\le i<j \le m(s,s')$ and $t_i=t_{i+m(s,s')}$. Hence, for $t \in T$, $n(\mathbf{s},t)=0$ or $2$. Applying the result in the previous paragraph, we see $U_{\mathbf{s}}=Id$. 

 Now we see $w\mapsto U_w$ is a homomorphism from $W$. Specifically, $U_w$ doesn't depend on the representation chosen for $w$. So from $(*)$, we see $n(\mathbf{s},t)$ is invariant among different choices of representations. This gives us $(1)$. $(2)$ follows easily. \qed

\vspace{\baselineskip}
\noindent \textbf{Theorem 1.9} For Coxeter System $(W,S)$, consider sequence $\mathbf{s}=(s_1,...,s_q)$, $\Phi(\mathbf{s})=(t_1,...,t_q)$, $w=s_1...s_q$. Then $\mathbf{s}$ is an reduced representation of $w$ if and only if $t_i \ne t_j$ for all $i \ne j$. Define $T_w:=\{t \in T \mid \eta(w,t)=-1\}$, then in case $s$ is a reduced representation of $w$, $T_w=\{t_1,...,t_q\}$ and $|T_w|=l(w)$.

\vspace{0.5\baselineskip} 
\noindent \textbf{Proof:} $\forall t \in T_w$,$n(w,t)\ne0$. By definition of $n(w,t)$, we deduce $\forall t \in T_w$,$t\in\{t_1,...,t_q\}$. Moreover, as $\eta(w,t)$ is independent of the choice of representation of $w$, $T_w$ only depends on $w$. Hence $|T_w|\le l(w)$. In case $t_i \ne t_j$ for all $i\ne j$, $n(s,t)=0$ or $1$, and hence $T_w=\{t_1,...,t_q\}$. As $q=|T_w| \le l(w)$, so $\mathbf{s}$ is a reduced representation of $w$. 

Conversely, if there exist $i \ne j$ such that $t_i=t_j$, WLOG say $i < j$. Consider $u=s_{i+1}...s_{j-1}$, then $s_i=us_ju^{-1}$. Hence we have $s_1...s_q=s_1...s_{i-1}(us_ju^{-1})us_js_{j+1}...s_q=s_1...s_{i-1}us_js_js_{j+1}...s_q=s_1...s_{i-1}s_{i+1}...s_{j-1}s_{j+1}...s_q$. So $\mathbf{s}$ is not reduced.\qed

\vspace{\baselineskip}
Note that now we have a necessary and sufficient condition to check whether a word is reduced or not. We want to improve this result and get an algorithm of reducing words. 

\vspace{\baselineskip}
\noindent \textbf{Definition 1.10} For a system $(W,S)$, if it satisfies: $\forall w \in W,s\in S$, if $l(sw) \le l(w)$, then for any reduced representation $\mathbf{s}=(s_1,...,s_q)$, exist $1\le j \le q$ such that $ss_1...s_{j-1}=s_1...s_j$. Then we say $(W,S)$ satisfies the exchange condition.

\vspace{\baselineskip}
\noindent \textbf{Theorem 1.11} If $(W,S)$ is a Coxeter System, then it satisfies the exchange condition. 

\vspace{0.5\baselineskip}
\noindent \textbf{Proof:} Consider $w\in W,s \in S$ such that $l(sw) \le l(w)$. For any reduced representation $\mathbf{s}=(s_1,...,s_q)$ of $w$, consider $w':=sw$. By Example 1.6, we have $l(w') \equiv l(w)+1\pmod{2}$. Proposition 1.3(3) gives that $|l(w)-l(w')|\le1$, so $l(w')=l(w)-1$. Now pick $(s_1',...,s'_{l(w)-1})$ as a reduced representation of $w'$, then $\mathbf{s'}=(s,s_1',...,s'_{p-1})$ is a reduced representation of $w$. Take $\Phi(\mathbf{s'})=(t_1',...,t'_p)$, then $t_1'=s$ by definition. However Theorem 1.9 shows that $t'_i \ne t'_j$ if $i\ne j$, so $n(\mathbf{s'},s)=1$. By Lemma 1.8, $n(\mathbf{s'},s) \equiv n(\mathbf{s},s)\pmod{2}$, so $n(\mathbf{s},s) \ne 0$. Hence there exist $j$ such that $s=t_j$, where $t_j$ is an element in $\Phi(\mathbf{s})$. By definition of $t_j$, $ss_1...s_{j-1}=s_1...s_j$, the claim is proved.\qed

\vspace{\baselineskip}
Theorem 1.11 is all we need to prove the Tit's theorem. But we can actually say more on exchange condition, and deduce conversely that any system such that $\forall s\in S$, $s$ has order 2 and satisfy exchange condition is a Coxeter System. We will talk more on it in part 4.

\subsection{M-operations}
\noindent \textbf{Definition 2.1} Consider $(W,S)$ a system, the sequence $\mathbf{s}=(s_1,...,s_q)$, where $s_i \in S,\forall i$ is called a word in $S$ and $w:=s_1...s_q$ is called the element in $W$ expressed by $\mathbf{s}$. An elementary M-operation on a word in $\mathbf{s}$ is one of the following two types of operations:

(MI) Delete a subword of form $(s,s)$ from $\mathbf{s}$.

(MII) Replace an alternating subword $(s,t,s,...)$ by the other alternating subword $(t,s,t,...)$, both in length $m(s,t)$.

We say a word is M-reduced if and only if its length cannot be shorten by M-operations. Clearly, any reduced word is M-reduced.

\vspace{\baselineskip}
\noindent \textbf{Lemma 2.2} For a system $(W,S)$ with Exchange condition, two reduced representations express the same element in $W$ if and only if one can be transformed to the other by MII operations.

\vspace{0.5\baselineskip} 
\noindent \textbf{Proof:} Say $\mathbf{s}=(s_1,...,s_q)$ and $\mathbf{r}=(r_1,...,r_q)$ two reduced representations both express $w \in W$. We prove the lemma by induction. 

In case $q=0$,the proof is trivial. If $s_1=r_1$, we can reduce the length of both words by 1 so the induction hypothesis can be applied directly. It suffice to prove the case $s_1 \ne r_1$. We set $m:=m(s_1,r_1)$.

\vspace{0.5\baselineskip}
\noindent \textbf{Claim:} m is finite and there is another reduced expression $\mathbf{u}$ of $w$ start with an alternating subword $(s_1,r_1,s_1,...)$ of length $m$.

\vspace{0.5\baselineskip}
With the claim, the remaining proof is easy. Since $\mathbf{s}$ and $\mathbf{u}$ both start with $s_1$, by induction hypothesis, $\mathbf{s}$ can be transformed into $\mathbf{u}$ by MII operations. Now apply MII on $\mathbf{u}$ to get another word $\mathbf{u'}$ starts with $(r_1,s_1,r_1,...)$ of length $m$. As $\mathbf{u'}$ and $\mathbf{r}$ both start with $r_1$, again by induction hypothesis, $\mathbf{u'}$ can be transformed into $\mathbf{r}$ through MII operations. The combination of all above MII operations gives the result. 

The converse of the proposition is trivial.\qed

\vspace{\baselineskip}
\noindent \textbf{Proof of the Claim:} Clearly, $l(r_1w) < l(w)$ as $\mathbf{r}$ is a reduced representation of $w$ starting with $r_1$. By the exchange condition there exist an index $i$ such that $r_1s_1...s_{i-1}=s_1...s_i$. Moreover, as $(s_1,...,s_i)$ is a subword of $\mathbf{s}$, which is reduced, $v:=(r_1,s_1,...,s_{i-1},s_{i+1},...,s_q)$ is also a reduced representation of $w$. 

\textit{Note that $i \ne 1$. Otherwise $(s_1,s_2,...,s_q)=(r_1,s_2,...,s_q)$ gives arise to $s_1=r_1$, which is a contradiction. }

Define $S_q$ to be the alternating word end in $s_1$ with each of its term in set $\{s_1,r_1\}$ of length $q$. Suppose $\mathbf{s'}$ a word start with $S_{q-1}$ and let $s'$ to be the element of $\{s_1,r_1\}$ that $\mathbf{s'}$ doesn't start with. 

Looking at $\mathbf{s}$ and $\mathbf{r}$, we get $l(s'w)<l(w)$. Apply the exchange condition we get a reduced representation of $\mathbf{s'}$ start with $s'$. The removed element through exchanging process should not lie in $S_{q-1}$ as any reduced representation with length not equal to $m$ an element in group $D_{2m}$ is unique. This is a fact we learned in Year 2 module Groups and Rings. If the removed term lies in $S_{q-1}$, we get two different reduce representations of element in $D_{2m}$ generated by $s_1,r_1$, length not equal to $m$ after cancellation, which is a contradiction. 

We hence get a reduced representation start with $S_q$ from $\mathbf{s'}$. Hence by induction, $w$ has a reduced expression start with $S_q$ for any $q \le m(s_1,r_1)$ but we have $q \le l(w) < \infty$. So we have $m$ is finite. 

Replace $S_{m(s_1,r_1)}$ with $(s_1,r_1,s_1,...)$ by MII operation, we get the expecting $\mathbf{u}$. \qed 

\subsection{Tit's Theorem}
\noindent \textbf{Theorem 3.1 (Tit's)} A word in system $(W,S)$ with exchange condition is reduced if and only if it's M-reduced. 

\vspace{0.5\baselineskip}
\noindent \textbf{Proof:} Suppose the word $\mathbf{s}=(s_1,...,s_k)$ is M-reduced. We show $\mathbf{s}$ is reduced by induction on $k$. The base case $k=1$ is trivial. 

For $k>1$, by induction hypothesis, $\mathbf{s'}=(s_2,...,s_k)$ is reduced. Say $w'$ the element expressed by $\mathbf{s'}$. Suppose $\mathbf{s}$ is not reduced, then $l(w)=l(s_1\mathbf{s'})\le k-1 < l(\mathbf{s'})$. So by exchange condition, there exist an index $i$ such that $w'=s_1s_2...s_{i-1}s_{i+1}...s_k$. That is, $w'$ has a reduced representation start with $s_1$. Apply Lemma 2.2 we see that $\mathbf{s''}:=(s_1,s_2,...,s_{i-1},s_{i+1},...,s_k)$ can be transformed from $\mathbf{s'}$ by MII operation. This gives a reduced representation of $w$ start with $(s_1,s_1)$, which is a contradiction. 

The other direction of the theorem is obvious.\qed

\vspace{\baselineskip}
Note that Tit's Theorem is far from trivial: The word in a free group can be really messy, even talk about $G=\langle x,y \mid x^n=y^2=(xy)^2=1_G \rangle$ directly is not easy, see Groups and Rings module of Year 2. 

\vspace{\baselineskip}
\noindent \textbf{Example 3.1} $G:=\langle x,y \mid xyxyx=yxyxy \rangle \cong \langle x,y,a \mid xyxyx=yxyxy,a=xy\rangle \cong \langle x,a \mid aax=x^{-1} aaa\rangle \cong \langle x,a,b \mid a^2x=x^{-1}a^3,b=xa^2\rangle \cong \langle a,b \mid a^2=b^5\rangle$

We can see from the example that although the relation of $G$ is clear, what operations can be done to make reduction is hard to see. Nontrivial process is done here to transfer the group of $x,y$ to a group of $a,b$ and enable us to do reduction.

There are general tools like Tietze Transformation and Coset enumeration to talk about what we can do on a general group presentation, but it's out of the scope of this project.

\section{More on Exchange Condition}
\noindent \textbf{Proposition 4.1} For system $(W,S)$ with exchange condition such that every element in $S$ has order 2, consider $w\in W$,$s\in S$. If $\mathbf{s}$ is a reduced representation of $w$, exactly one of the following occurs:

(1) $l(sw)=l(w)+1$,$(s,s_1,...,s_q)$ is a reduced representation of $sw$.
(2) $l(sw)=l(w)-1$,$\exists 1\le j \le q$  such that $(s_1,...,s_{j-1},s_{j+1},...,s_q)$ is a reduced representation of $sw$ and $(s,s_1,...,s_{j-1},s_{j+1},...,s_q)$ is a reduce representation of $w$. 

\noindent \textbf{Proof:} This is a direct application of exchange condition.

\end{document}